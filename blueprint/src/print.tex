% This file makes a printable version of the blueprint
% It should include all the \usepackage needed for the pdf version.
% The template version assume you want to use a modern TeX compiler
% such as xeLaTeX or luaLaTeX including support for unicode
% and Latin Modern Math font with standard bugfixes applied.
% It also uses expl3 in order to support macros related to the dependency graph.
% It also includes standard AMS packages (and their improved version
% mathtools) as well as support for links with a sober decoration
% (no ugly rectangles around links).
% It is otherwise a very minimal preamble (you should probably at least
% add cleveref and tikz-cd).

\documentclass[a4paper]{report}

\usepackage{geometry}

\usepackage{expl3}

\usepackage{amssymb, amsthm, mathtools, lineno}
\usepackage[unicode,colorlinks=true,linkcolor=blue,urlcolor=magenta, citecolor=blue]{hyperref}

\usepackage[warnings-off={mathtools-colon,mathtools-overbracket}]{unicode-math}
\usepackage{enumitem}

% In this file you should put all LaTeX macros and settings to be used both by
% the pdf version and the web version.
% This should be most of your macros.

% The theorem-like environments defined below are those that appear by default
% in the dependency graph. See the README of leanblueprint if you need help to
% customize this.
% The configuration below use the theorem counter for all those environments
% (this is what the [theorem] arguments mean) and never resets it.
% If you want for instance to number them within chapters then you can add
% [chapter] at the end of the next line.

% \newtheorem{theorem}{Theorem}
% \newtheorem{proposition}[theorem]{Proposition}
% \newtheorem{lemma}[theorem]{Lemma}
% \newtheorem{corollary}[theorem]{Corollary}

% \theoremstyle{definition}
% \newtheorem{definition}[theorem]{Definition}

% Define the Cyrillic letter "che" (ч) using a font that supports it
%\newfontfamily{\cyrillicfont}{Times New Roman}
\usepackage{fontspec}
\setmainfont{TeX Gyre Termes}
\newfontfamily\cyrillicfont{TeX Gyre Termes}[Script=Cyrillic]

\newcommand{\letterche}{\text{{\cyrillicfont ч}}}

\newtheorem{cor}{Corollary}
\newtheorem{thm}[cor]{Theorem}
\newtheorem{lem}[cor]{Lemma}
\newtheorem{prop}[cor]{Proposition}

\theoremstyle{remark}
\newtheorem{rem}[cor]{Remark}

\newcommand{\C}{\Omega}
\newcommand{\Cn}{\C_{n}(a)}
\newcommand{\CheOrd}{\letterche}
\newcommand{\Che}{\CheOrd_{p}}
\newcommand{\N}{{\mathbb{N}}}
\newcommand{\Z}{{\mathbb{Z}}}
\newcommand{\Q}{{\mathbb{Q}}}
\newcommand{\ord}{{\mathrm{ord }}}
\newcommand{\supp}{{\mathrm{supp}}}

% This file makes a printable version of the blueprint
% It should include all the \usepackage needed for the pdf version.
% The template version assume you want to use a modern TeX compiler
% such as xeLaTeX or luaLaTeX including support for unicode
% and Latin Modern Math font with standard bugfixes applied.
% It also uses expl3 in order to support macros related to the dependency graph.
% It also includes standard AMS packages (and their improved version
% mathtools) as well as support for links with a sober decoration
% (no ugly rectangles around links).
% It is otherwise a very minimal preamble (you should probably at least
% add cleveref and tikz-cd).

\documentclass[a4paper]{report}

\usepackage{geometry}

\usepackage{expl3}

\usepackage{amssymb, amsthm, mathtools, lineno}
\usepackage[unicode,colorlinks=true,linkcolor=blue,urlcolor=magenta, citecolor=blue]{hyperref}

\usepackage[warnings-off={mathtools-colon,mathtools-overbracket}]{unicode-math}
\usepackage{enumitem}

% In this file you should put all LaTeX macros and settings to be used both by
% the pdf version and the web version.
% This should be most of your macros.

% The theorem-like environments defined below are those that appear by default
% in the dependency graph. See the README of leanblueprint if you need help to
% customize this.
% The configuration below use the theorem counter for all those environments
% (this is what the [theorem] arguments mean) and never resets it.
% If you want for instance to number them within chapters then you can add
% [chapter] at the end of the next line.

% \newtheorem{theorem}{Theorem}
% \newtheorem{proposition}[theorem]{Proposition}
% \newtheorem{lemma}[theorem]{Lemma}
% \newtheorem{corollary}[theorem]{Corollary}

% \theoremstyle{definition}
% \newtheorem{definition}[theorem]{Definition}

% Define the Cyrillic letter "che" (ч) using a font that supports it
%\newfontfamily{\cyrillicfont}{Times New Roman}
\usepackage{fontspec}
\setmainfont{TeX Gyre Termes}
\newfontfamily\cyrillicfont{TeX Gyre Termes}[Script=Cyrillic]

\newcommand{\letterche}{\text{{\cyrillicfont ч}}}

\newtheorem{cor}{Corollary}
\newtheorem{thm}[cor]{Theorem}
\newtheorem{lem}[cor]{Lemma}
\newtheorem{prop}[cor]{Proposition}

\theoremstyle{remark}
\newtheorem{rem}[cor]{Remark}

\newcommand{\C}{\Omega}
\newcommand{\Cn}{\C_{n}(a)}
\newcommand{\CheOrd}{\letterche}
\newcommand{\Che}{\CheOrd_{p}}
\newcommand{\N}{{\mathbb{N}}}
\newcommand{\Z}{{\mathbb{Z}}}
\newcommand{\Q}{{\mathbb{Q}}}
\newcommand{\ord}{{\mathrm{ord }}}
\newcommand{\supp}{{\mathrm{supp}}}

% This file makes a printable version of the blueprint
% It should include all the \usepackage needed for the pdf version.
% The template version assume you want to use a modern TeX compiler
% such as xeLaTeX or luaLaTeX including support for unicode
% and Latin Modern Math font with standard bugfixes applied.
% It also uses expl3 in order to support macros related to the dependency graph.
% It also includes standard AMS packages (and their improved version
% mathtools) as well as support for links with a sober decoration
% (no ugly rectangles around links).
% It is otherwise a very minimal preamble (you should probably at least
% add cleveref and tikz-cd).

\documentclass[a4paper]{report}

\usepackage{geometry}

\usepackage{expl3}

\usepackage{amssymb, amsthm, mathtools, lineno}
\usepackage[unicode,colorlinks=true,linkcolor=blue,urlcolor=magenta, citecolor=blue]{hyperref}

\usepackage[warnings-off={mathtools-colon,mathtools-overbracket}]{unicode-math}
\usepackage{enumitem}

% In this file you should put all LaTeX macros and settings to be used both by
% the pdf version and the web version.
% This should be most of your macros.

% The theorem-like environments defined below are those that appear by default
% in the dependency graph. See the README of leanblueprint if you need help to
% customize this.
% The configuration below use the theorem counter for all those environments
% (this is what the [theorem] arguments mean) and never resets it.
% If you want for instance to number them within chapters then you can add
% [chapter] at the end of the next line.

% \newtheorem{theorem}{Theorem}
% \newtheorem{proposition}[theorem]{Proposition}
% \newtheorem{lemma}[theorem]{Lemma}
% \newtheorem{corollary}[theorem]{Corollary}

% \theoremstyle{definition}
% \newtheorem{definition}[theorem]{Definition}

% Define the Cyrillic letter "che" (ч) using a font that supports it
%\newfontfamily{\cyrillicfont}{Times New Roman}
\usepackage{fontspec}
\setmainfont{TeX Gyre Termes}
\newfontfamily\cyrillicfont{TeX Gyre Termes}[Script=Cyrillic]

\newcommand{\letterche}{\text{{\cyrillicfont ч}}}

\newtheorem{cor}{Corollary}
\newtheorem{thm}[cor]{Theorem}
\newtheorem{lem}[cor]{Lemma}
\newtheorem{prop}[cor]{Proposition}

\theoremstyle{remark}
\newtheorem{rem}[cor]{Remark}

\newcommand{\C}{\Omega}
\newcommand{\Cn}{\C_{n}(a)}
\newcommand{\CheOrd}{\letterche}
\newcommand{\Che}{\CheOrd_{p}}
\newcommand{\N}{{\mathbb{N}}}
\newcommand{\Z}{{\mathbb{Z}}}
\newcommand{\Q}{{\mathbb{Q}}}
\newcommand{\ord}{{\mathrm{ord }}}
\newcommand{\supp}{{\mathrm{supp}}}

% This file makes a printable version of the blueprint
% It should include all the \usepackage needed for the pdf version.
% The template version assume you want to use a modern TeX compiler
% such as xeLaTeX or luaLaTeX including support for unicode
% and Latin Modern Math font with standard bugfixes applied.
% It also uses expl3 in order to support macros related to the dependency graph.
% It also includes standard AMS packages (and their improved version
% mathtools) as well as support for links with a sober decoration
% (no ugly rectangles around links).
% It is otherwise a very minimal preamble (you should probably at least
% add cleveref and tikz-cd).

\documentclass[a4paper]{report}

\usepackage{geometry}

\usepackage{expl3}

\usepackage{amssymb, amsthm, mathtools, lineno}
\usepackage[unicode,colorlinks=true,linkcolor=blue,urlcolor=magenta, citecolor=blue]{hyperref}

\usepackage[warnings-off={mathtools-colon,mathtools-overbracket}]{unicode-math}
\usepackage{enumitem}

\input{macros/common}
\input{macros/print}

\title{Zsigmondy's Theorem for Chebyshev Polynomials}
\author{Stefan Barańczuk (formalized by Bartosz Naskręcki via Aristotle)}

\begin{document}
\maketitle
\begin{abstract}
  For an integer $a$ consider the divisibility sequence $s_{n}=T_{n}(a)-1$ defined by the
  Chebyshev polynomials $T_{n}$. We list all values of $a$ and $n$ for which the
  term $s_{n}$ has no primitive prime divisor. This result is a pendant to the classic
  Zsigmondy's Theorem, an analogous result for power maps.
\end{abstract}
\input{content}
\end{document}


\title{Zsigmondy's Theorem for Chebyshev Polynomials}
\author{Stefan Barańczuk (formalized by Bartosz Naskręcki via Aristotle)}

\begin{document}
\maketitle
\begin{abstract}
  For an integer $a$ consider the divisibility sequence $s_{n}=T_{n}(a)-1$ defined by the
  Chebyshev polynomials $T_{n}$. We list all values of $a$ and $n$ for which the
  term $s_{n}$ has no primitive prime divisor. This result is a pendant to the classic
  Zsigmondy's Theorem, an analogous result for power maps.
\end{abstract}
% In this file you should put the actual content of the blueprint.
% It will be used both by the web and the print version.
% It should *not* include the \begin{document}
%
% If you want to split the blueprint content into several files then
% the current file can be a simple sequence of \input. Otherwise It
% can start with a \section or \chapter for instance.

\chapter{Introduction}

Among all polynomial sequences, the following two seem to be most distinguished: the sequence of the
power maps $x^{n}$ and the sequence of the Chebyshev polynomials $T_{n}(x)$, defined either by the
property $T_{n}(\cos(\theta))=\cos(n\theta)$, or recursively
$T_{0}(x)=1$, $T_{1}(x)=x$, $T_{n+2}(x)=2xT_{n+1}(x)-T_{n}(x)$.

Their best known mutual property is probably the composition identity, which we state here for the
Chebyshev polynomials: \[T_{n}(T_{m}(x))=T_{m}(T_{n}(x))=T_{mn}(x).\]

In this paper we investigate such property -- namely, we prove the Chebyshev polynomials analogue
of Zsigmondy's Theorem.

Zsigmondy's Theorem says for which natural numbers $a,n>1$ there is a prime divisor $p$ of $a^{n}-1$
that does not divide any of the numbers $a^{d}-1$,  $d<n$ (such primes are called
\textit{primitive prime divisors}) or, equivalently, there is a prime number $p$ such that the
multiplicative order $\ord_{p}(a)$ equals $n$.

The above mentioned link between the power maps and the Chebyshev polynomials evoke the question
whether we could replace $a^{n}$ in Zsigmondy's Theorem by $T_{n}(a)$. Our answer is
Theorem \ref{Satz-2}; in order to formulate it we have to introduce the following Chebyshevian analogue
of the multiplicative order. For an integer $a$ and a prime number $p$ denote
\footnote{Since most Latin and Greek characters are already used in number theory notation, we find
the choice of Cyrillic character $\letterche$ (the lowercase of the letter Che) to be appropriate.}
\[\Che (a)=\min \left\lbrace m\in\Z_{>0}\, \colon \, T_{m}(a) \equiv  1 \mod p \right\rbrace ;\]
this quantity always exists by Lemma \ref{LFT}.

\begin{theorem}\label{Satz-2}
	\lean{Satz_2}
	\leanok
	Let $n$ and $a$  be integers such that $n>0$.  There exists a prime number $p$ such
	that $n=\Che(a)$, except in the following cases:
	\begin{itemize}
		\item $a=1$ and $n>1$,
		\item $a=0$ and $n\notin\left\lbrace 2,4\right\rbrace $,
		\item $a=-1$ and $n>2$,
		\item $n=1$ and $a\in \left\lbrace0,2 \right\rbrace $,
		\item $n=2$ and $a=\pm 2^{\alpha-1}-1$,
		\item $n=3$ and $a=\tfrac{1}{2}(\pm 3^{\alpha}-1)$,
		\item $n=4$ and $a=\pm 2^{\alpha-1}$,
		\item $n=6$ and $a=\tfrac{1}{2}(\pm 3^{\alpha}+1)$,
	\end{itemize}
	where $\alpha$ runs through positive integers.
\end{theorem}

\chapter{Preliminary results}
\begin{proposition}\label{RivExer}
	\lean{RivExer}
	\leanok
	If $a,b$ are nonnegative integers, then
	\[(T_{a+b}(x)-1)(T_{\left|a-b \right| }(x)-1)=(T_{a}(x)-T_{b}(x))^2.\]
\end{proposition}

The following lemma is the analogue of Fermat's little theorem for Chebyshev polynomials.

\begin{lemma}\label{LFT}
	\lean{LFT}
	\leanok
	Let $p$ be an odd prime number. For every integer $x$
	\[ T_{p-1}(x)\equiv 1 \mod p \,\,\, \text{or} \,\,\, T_{p+1}(x)\equiv 1 \mod p.\]
	For every integer $x$
	\[ T_{2}(x)\equiv 1 \mod 2. \]
\end{lemma}

\begin{lemma}\label{ChebOrd}
	\lean{ChebOrd}
	\leanok
	Let $p$ be a prime number and $x$ be an integer. Let $m$ be the smallest positive integer
	such that $T_{m}(x) \equiv  1 \mod p$.
	Then $T_{n}(x) \equiv  1 \mod p$ for a positive integer $n$ if and only if $m \mid n$.
\end{lemma}

Lemmas \ref{LFT} and \ref{ChebOrd} together with $T_{1}(x)=x$  give the following.

\begin{lemma}\label{prop-Che}
	\lean{prop_Che}
	\leanok
	If $x$ is an integer and $p$ is an odd prime number then $\Che (x)$ divides $p-1$ or $p+1$.
	In particular, $\Che (x)$ and $p$ are coprime.
	If $x$ is odd then $\CheOrd_{2}(x)=1$. If $x$ is even then $\CheOrd_{2}(x)=2$.
\end{lemma}

\begin{lemma}\label{inrtocyc}
	\lean{inrtocyc}
	\leanok
	For every $n\ge 1$
	\[T_{n}(x)-1=\prod_{d \mid n}\C_{d}^{\sigma_{d}}(x)\]
	where $\C_{1}(x)=x-1$ and for $d\ge 2$
	\[\C_{d}(x)=\prod_{
		\substack{
			1\le k \le \frac{d}{2}\\
			\gcd(k,d)=1}}2(x-\cos\tfrac{2 k \pi}{d})\]
	and \[\sigma_{d}= \left\{
	\begin{array}{rl}
		1 & \text{if } d=1,2,\\
		2 & \text{if } d >2.
	\end{array} \right.\]
\end{lemma}

\begin{proposition}\label{omegas-prop-b}
	\lean{omegas_prop_b}
	\leanok
	Let $m, n$ be positive integers.
	Then the polynomial $\C_{mn}$ is a divisor of $\C_{n}(T_{m}(x))$.
	If moreover $n\ge 3$ and every prime divisor of $m$ divides $n$, then $\C_{mn}=\C_{n}(T_{m}(x))$.
\end{proposition}

\begin{proposition}\label{omega-odd}
	\lean{omega_odd}
	\leanok
	Let $n$ be an odd natural number. Then $\C_{n}(0)=\pm 1$.
	If moreover $n\ge 3$ then $\C_{2n}(x)=\pm\C_{n}(-x)$.
\end{proposition}

\begin{proposition}\label{poly-square-int}
	\lean{poly_square_int}
	\leanok
	Let $\mathbb{K}$ be a field of characteristic $0$. Suppose that $P\in \mathbb{K}[x]$, $P(0)=\pm 1$,
	and $P^{2}\in \Z[x]$. Then $P\in \Z[x]$.
\end{proposition}

\begin{lemma}\label{Omega-integral}
	\lean{Omega_integral}
	\leanok
	$\C_{n}\in \Z[x]$.
\end{lemma}

\begin{corollary}\label{if-T-then-Omega}
	\lean{if_T_then_Omega}
	\leanok
	Let $x$ be an integer.
	Every primitive prime divisor of $T_{n}(x)-1$ divides $\C_{n}(x)$.
\end{corollary}

\begin{proposition}\label{T-on-x-plus-one-minus-one-by-x}
	\lean{T_on_x_plus_one_minus_one_by_x}
	\leanok
	For every natural number $n$ and every nonzero real number $x$
	\[\frac{T_{n}(x+1)-1}{x}=n^2+\frac{n^2(n^2-1)}{6}x+\frac{n^2(n^2-1)(n^2-4)}{90}x^{2}+\ldots \in \Z[x],\]
	where the dots denote irrelevant terms.
\end{proposition}

\chapter{Major results on \texorpdfstring{$\C$}{C}'s}

\begin{theorem}\label{Satz-1}
	\lean{Satz_1_part1,Satz_1_part2, Satz_1_part3_ge_5, Satz_1_part3_p_eq_3_corrected,
		Satz_1_part3_p_eq_2_corrected}
	\leanok
	Let $n$ and $a$  be integers such that $n>1$ and $\left| a\right| >1$. If $p$ is a prime number
	dividing $\Cn$ then there exists a nonnegative integer $i$ such that $n=\Che(a)p^{i}$. If $i>0$,
	then $p$ is the greatest prime divisor of $n$. If moreover $p^2 \mid \Cn$ then either $p=2$ and
	$n\in\left\lbrace 2,4\right\rbrace$, or $p=3$ and $n\in\left\lbrace 3,6\right\rbrace $.
\end{theorem}

\begin{corollary}\label{wniosek}
	\lean{wniosek}
	\leanok
	Let $n$ and $a$  be integers such that $n>1$ and $\left| a\right| >1$. The following statements are equivalent.
	\begin{enumerate}[label={\text{S}\arabic*}]
		\item \label{TFAE1} A prime number $p$ such that $n=\Che(a)$ does not exist.
		\item \label{TFAE2} $\left| \Cn \right|$  is a power of a prime number dividing $n$.
		\item \label{TFAE3} $\left| \Cn \right|$  is a power of the greatest prime divisor of $n$.
	\end{enumerate}
\end{corollary}

\end{document}


\title{Zsigmondy's Theorem for Chebyshev Polynomials}
\author{Stefan Barańczuk (formalized by Bartosz Naskręcki via Aristotle)}

\begin{document}
\maketitle
\begin{abstract}
  For an integer $a$ consider the divisibility sequence $s_{n}=T_{n}(a)-1$ defined by the
  Chebyshev polynomials $T_{n}$. We list all values of $a$ and $n$ for which the
  term $s_{n}$ has no primitive prime divisor. This result is a pendant to the classic
  Zsigmondy's Theorem, an analogous result for power maps.
\end{abstract}
% In this file you should put the actual content of the blueprint.
% It will be used both by the web and the print version.
% It should *not* include the \begin{document}
%
% If you want to split the blueprint content into several files then
% the current file can be a simple sequence of \input. Otherwise It
% can start with a \section or \chapter for instance.

\chapter{Introduction}

Among all polynomial sequences, the following two seem to be most distinguished: the sequence of the
power maps $x^{n}$ and the sequence of the Chebyshev polynomials $T_{n}(x)$, defined either by the
property $T_{n}(\cos(\theta))=\cos(n\theta)$, or recursively
$T_{0}(x)=1$, $T_{1}(x)=x$, $T_{n+2}(x)=2xT_{n+1}(x)-T_{n}(x)$.

Their best known mutual property is probably the composition identity, which we state here for the
Chebyshev polynomials: \[T_{n}(T_{m}(x))=T_{m}(T_{n}(x))=T_{mn}(x).\]

In this paper we investigate such property -- namely, we prove the Chebyshev polynomials analogue
of Zsigmondy's Theorem.

Zsigmondy's Theorem says for which natural numbers $a,n>1$ there is a prime divisor $p$ of $a^{n}-1$
that does not divide any of the numbers $a^{d}-1$,  $d<n$ (such primes are called
\textit{primitive prime divisors}) or, equivalently, there is a prime number $p$ such that the
multiplicative order $\ord_{p}(a)$ equals $n$.

The above mentioned link between the power maps and the Chebyshev polynomials evoke the question
whether we could replace $a^{n}$ in Zsigmondy's Theorem by $T_{n}(a)$. Our answer is
Theorem \ref{Satz-2}; in order to formulate it we have to introduce the following Chebyshevian analogue
of the multiplicative order. For an integer $a$ and a prime number $p$ denote
\footnote{Since most Latin and Greek characters are already used in number theory notation, we find
the choice of Cyrillic character $\letterche$ (the lowercase of the letter Che) to be appropriate.}
\[\Che (a)=\min \left\lbrace m\in\Z_{>0}\, \colon \, T_{m}(a) \equiv  1 \mod p \right\rbrace ;\]
this quantity always exists by Lemma \ref{LFT}.

\begin{theorem}\label{Satz-2}
	\lean{Satz_2}
	\leanok
	Let $n$ and $a$  be integers such that $n>0$.  There exists a prime number $p$ such
	that $n=\Che(a)$, except in the following cases:
	\begin{itemize}
		\item $a=1$ and $n>1$,
		\item $a=0$ and $n\notin\left\lbrace 2,4\right\rbrace $,
		\item $a=-1$ and $n>2$,
		\item $n=1$ and $a\in \left\lbrace0,2 \right\rbrace $,
		\item $n=2$ and $a=\pm 2^{\alpha-1}-1$,
		\item $n=3$ and $a=\tfrac{1}{2}(\pm 3^{\alpha}-1)$,
		\item $n=4$ and $a=\pm 2^{\alpha-1}$,
		\item $n=6$ and $a=\tfrac{1}{2}(\pm 3^{\alpha}+1)$,
	\end{itemize}
	where $\alpha$ runs through positive integers.
\end{theorem}

\chapter{Preliminary results}
\begin{proposition}\label{RivExer}
	\lean{RivExer}
	\leanok
	If $a,b$ are nonnegative integers, then
	\[(T_{a+b}(x)-1)(T_{\left|a-b \right| }(x)-1)=(T_{a}(x)-T_{b}(x))^2.\]
\end{proposition}

The following lemma is the analogue of Fermat's little theorem for Chebyshev polynomials.

\begin{lemma}\label{LFT}
	\lean{LFT}
	\leanok
	Let $p$ be an odd prime number. For every integer $x$
	\[ T_{p-1}(x)\equiv 1 \mod p \,\,\, \text{or} \,\,\, T_{p+1}(x)\equiv 1 \mod p.\]
	For every integer $x$
	\[ T_{2}(x)\equiv 1 \mod 2. \]
\end{lemma}

\begin{lemma}\label{ChebOrd}
	\lean{ChebOrd}
	\leanok
	Let $p$ be a prime number and $x$ be an integer. Let $m$ be the smallest positive integer
	such that $T_{m}(x) \equiv  1 \mod p$.
	Then $T_{n}(x) \equiv  1 \mod p$ for a positive integer $n$ if and only if $m \mid n$.
\end{lemma}

Lemmas \ref{LFT} and \ref{ChebOrd} together with $T_{1}(x)=x$  give the following.

\begin{lemma}\label{prop-Che}
	\lean{prop_Che}
	\leanok
	If $x$ is an integer and $p$ is an odd prime number then $\Che (x)$ divides $p-1$ or $p+1$.
	In particular, $\Che (x)$ and $p$ are coprime.
	If $x$ is odd then $\CheOrd_{2}(x)=1$. If $x$ is even then $\CheOrd_{2}(x)=2$.
\end{lemma}

\begin{lemma}\label{inrtocyc}
	\lean{inrtocyc}
	\leanok
	For every $n\ge 1$
	\[T_{n}(x)-1=\prod_{d \mid n}\C_{d}^{\sigma_{d}}(x)\]
	where $\C_{1}(x)=x-1$ and for $d\ge 2$
	\[\C_{d}(x)=\prod_{
		\substack{
			1\le k \le \frac{d}{2}\\
			\gcd(k,d)=1}}2(x-\cos\tfrac{2 k \pi}{d})\]
	and \[\sigma_{d}= \left\{
	\begin{array}{rl}
		1 & \text{if } d=1,2,\\
		2 & \text{if } d >2.
	\end{array} \right.\]
\end{lemma}

\begin{proposition}\label{omegas-prop-b}
	\lean{omegas_prop_b}
	\leanok
	Let $m, n$ be positive integers.
	Then the polynomial $\C_{mn}$ is a divisor of $\C_{n}(T_{m}(x))$.
	If moreover $n\ge 3$ and every prime divisor of $m$ divides $n$, then $\C_{mn}=\C_{n}(T_{m}(x))$.
\end{proposition}

\begin{proposition}\label{omega-odd}
	\lean{omega_odd}
	\leanok
	Let $n$ be an odd natural number. Then $\C_{n}(0)=\pm 1$.
	If moreover $n\ge 3$ then $\C_{2n}(x)=\pm\C_{n}(-x)$.
\end{proposition}

\begin{proposition}\label{poly-square-int}
	\lean{poly_square_int}
	\leanok
	Let $\mathbb{K}$ be a field of characteristic $0$. Suppose that $P\in \mathbb{K}[x]$, $P(0)=\pm 1$,
	and $P^{2}\in \Z[x]$. Then $P\in \Z[x]$.
\end{proposition}

\begin{lemma}\label{Omega-integral}
	\lean{Omega_integral}
	\leanok
	$\C_{n}\in \Z[x]$.
\end{lemma}

\begin{corollary}\label{if-T-then-Omega}
	\lean{if_T_then_Omega}
	\leanok
	Let $x$ be an integer.
	Every primitive prime divisor of $T_{n}(x)-1$ divides $\C_{n}(x)$.
\end{corollary}

\begin{proposition}\label{T-on-x-plus-one-minus-one-by-x}
	\lean{T_on_x_plus_one_minus_one_by_x}
	\leanok
	For every natural number $n$ and every nonzero real number $x$
	\[\frac{T_{n}(x+1)-1}{x}=n^2+\frac{n^2(n^2-1)}{6}x+\frac{n^2(n^2-1)(n^2-4)}{90}x^{2}+\ldots \in \Z[x],\]
	where the dots denote irrelevant terms.
\end{proposition}

\chapter{Major results on \texorpdfstring{$\C$}{C}'s}

\begin{theorem}\label{Satz-1}
	\lean{Satz_1_part1,Satz_1_part2, Satz_1_part3_ge_5, Satz_1_part3_p_eq_3_corrected,
		Satz_1_part3_p_eq_2_corrected}
	\leanok
	Let $n$ and $a$  be integers such that $n>1$ and $\left| a\right| >1$. If $p$ is a prime number
	dividing $\Cn$ then there exists a nonnegative integer $i$ such that $n=\Che(a)p^{i}$. If $i>0$,
	then $p$ is the greatest prime divisor of $n$. If moreover $p^2 \mid \Cn$ then either $p=2$ and
	$n\in\left\lbrace 2,4\right\rbrace$, or $p=3$ and $n\in\left\lbrace 3,6\right\rbrace $.
\end{theorem}

\begin{corollary}\label{wniosek}
	\lean{wniosek}
	\leanok
	Let $n$ and $a$  be integers such that $n>1$ and $\left| a\right| >1$. The following statements are equivalent.
	\begin{enumerate}[label={\text{S}\arabic*}]
		\item \label{TFAE1} A prime number $p$ such that $n=\Che(a)$ does not exist.
		\item \label{TFAE2} $\left| \Cn \right|$  is a power of a prime number dividing $n$.
		\item \label{TFAE3} $\left| \Cn \right|$  is a power of the greatest prime divisor of $n$.
	\end{enumerate}
\end{corollary}

\end{document}


\title{Zsigmondy's Theorem for Chebyshev Polynomials}
\author{Stefan Barańczuk (formalized by Bartosz Naskręcki via Aristotle)}

\begin{document}
\maketitle
\begin{abstract}
  For an integer $a$ consider the divisibility sequence $s_{n}=T_{n}(a)-1$ defined by the
  Chebyshev polynomials $T_{n}$. We list all values of $a$ and $n$ for which the
  term $s_{n}$ has no primitive prime divisor. This result is a pendant to the classic
  Zsigmondy's Theorem, an analogous result for power maps.
\end{abstract}
% In this file you should put the actual content of the blueprint.
% It will be used both by the web and the print version.
% It should *not* include the \begin{document}
%
% If you want to split the blueprint content into several files then
% the current file can be a simple sequence of \input. Otherwise It
% can start with a \section or \chapter for instance.

\chapter{Introduction}

Among all polynomial sequences, the following two seem to be most distinguished: the sequence of the
power maps $x^{n}$ and the sequence of the Chebyshev polynomials $T_{n}(x)$, defined either by the
property $T_{n}(\cos(\theta))=\cos(n\theta)$, or recursively
$T_{0}(x)=1$, $T_{1}(x)=x$, $T_{n+2}(x)=2xT_{n+1}(x)-T_{n}(x)$.

Their best known mutual property is probably the composition identity, which we state here for the
Chebyshev polynomials: \[T_{n}(T_{m}(x))=T_{m}(T_{n}(x))=T_{mn}(x).\]

In this paper we investigate such property -- namely, we prove the Chebyshev polynomials analogue
of Zsigmondy's Theorem.

Zsigmondy's Theorem says for which natural numbers $a,n>1$ there is a prime divisor $p$ of $a^{n}-1$
that does not divide any of the numbers $a^{d}-1$,  $d<n$ (such primes are called
\textit{primitive prime divisors}) or, equivalently, there is a prime number $p$ such that the
multiplicative order $\ord_{p}(a)$ equals $n$.

The above mentioned link between the power maps and the Chebyshev polynomials evoke the question
whether we could replace $a^{n}$ in Zsigmondy's Theorem by $T_{n}(a)$. Our answer is
Theorem \ref{Satz-2}; in order to formulate it we have to introduce the following Chebyshevian analogue
of the multiplicative order. For an integer $a$ and a prime number $p$ denote
\footnote{Since most Latin and Greek characters are already used in number theory notation, we find
the choice of Cyrillic character $\letterche$ (the lowercase of the letter Che) to be appropriate.}
\[\Che (a)=\min \left\lbrace m\in\Z_{>0}\, \colon \, T_{m}(a) \equiv  1 \mod p \right\rbrace ;\]
this quantity always exists by Lemma \ref{LFT}.

\begin{theorem}\label{Satz-2}
	\lean{Satz_2}
	\leanok
	Let $n$ and $a$  be integers such that $n>0$.  There exists a prime number $p$ such
	that $n=\Che(a)$, except in the following cases:
	\begin{itemize}
		\item $a=1$ and $n>1$,
		\item $a=0$ and $n\notin\left\lbrace 2,4\right\rbrace $,
		\item $a=-1$ and $n>2$,
		\item $n=1$ and $a\in \left\lbrace0,2 \right\rbrace $,
		\item $n=2$ and $a=\pm 2^{\alpha-1}-1$,
		\item $n=3$ and $a=\tfrac{1}{2}(\pm 3^{\alpha}-1)$,
		\item $n=4$ and $a=\pm 2^{\alpha-1}$,
		\item $n=6$ and $a=\tfrac{1}{2}(\pm 3^{\alpha}+1)$,
	\end{itemize}
	where $\alpha$ runs through positive integers.
\end{theorem}

\chapter{Preliminary results}
\begin{proposition}\label{RivExer}
	\lean{RivExer}
	\leanok
	If $a,b$ are nonnegative integers, then
	\[(T_{a+b}(x)-1)(T_{\left|a-b \right| }(x)-1)=(T_{a}(x)-T_{b}(x))^2.\]
\end{proposition}

The following lemma is the analogue of Fermat's little theorem for Chebyshev polynomials.

\begin{lemma}\label{LFT}
	\lean{LFT}
	\leanok
	Let $p$ be an odd prime number. For every integer $x$
	\[ T_{p-1}(x)\equiv 1 \mod p \,\,\, \text{or} \,\,\, T_{p+1}(x)\equiv 1 \mod p.\]
	For every integer $x$
	\[ T_{2}(x)\equiv 1 \mod 2. \]
\end{lemma}

\begin{lemma}\label{ChebOrd}
	\lean{ChebOrd}
	\leanok
	Let $p$ be a prime number and $x$ be an integer. Let $m$ be the smallest positive integer
	such that $T_{m}(x) \equiv  1 \mod p$.
	Then $T_{n}(x) \equiv  1 \mod p$ for a positive integer $n$ if and only if $m \mid n$.
\end{lemma}

Lemmas \ref{LFT} and \ref{ChebOrd} together with $T_{1}(x)=x$  give the following.

\begin{lemma}\label{prop-Che}
	\lean{prop_Che}
	\leanok
	If $x$ is an integer and $p$ is an odd prime number then $\Che (x)$ divides $p-1$ or $p+1$.
	In particular, $\Che (x)$ and $p$ are coprime.
	If $x$ is odd then $\CheOrd_{2}(x)=1$. If $x$ is even then $\CheOrd_{2}(x)=2$.
\end{lemma}

\begin{lemma}\label{inrtocyc}
	\lean{inrtocyc}
	\leanok
	For every $n\ge 1$
	\[T_{n}(x)-1=\prod_{d \mid n}\C_{d}^{\sigma_{d}}(x)\]
	where $\C_{1}(x)=x-1$ and for $d\ge 2$
	\[\C_{d}(x)=\prod_{
		\substack{
			1\le k \le \frac{d}{2}\\
			\gcd(k,d)=1}}2(x-\cos\tfrac{2 k \pi}{d})\]
	and \[\sigma_{d}= \left\{
	\begin{array}{rl}
		1 & \text{if } d=1,2,\\
		2 & \text{if } d >2.
	\end{array} \right.\]
\end{lemma}

\begin{proposition}\label{omegas-prop-b}
	\lean{omegas_prop_b}
	\leanok
	Let $m, n$ be positive integers.
	Then the polynomial $\C_{mn}$ is a divisor of $\C_{n}(T_{m}(x))$.
	If moreover $n\ge 3$ and every prime divisor of $m$ divides $n$, then $\C_{mn}=\C_{n}(T_{m}(x))$.
\end{proposition}

\begin{proposition}\label{omega-odd}
	\lean{omega_odd}
	\leanok
	Let $n$ be an odd natural number. Then $\C_{n}(0)=\pm 1$.
	If moreover $n\ge 3$ then $\C_{2n}(x)=\pm\C_{n}(-x)$.
\end{proposition}

\begin{proposition}\label{poly-square-int}
	\lean{poly_square_int}
	\leanok
	Let $\mathbb{K}$ be a field of characteristic $0$. Suppose that $P\in \mathbb{K}[x]$, $P(0)=\pm 1$,
	and $P^{2}\in \Z[x]$. Then $P\in \Z[x]$.
\end{proposition}

\begin{lemma}\label{Omega-integral}
	\lean{Omega_integral}
	\leanok
	$\C_{n}\in \Z[x]$.
\end{lemma}

\begin{corollary}\label{if-T-then-Omega}
	\lean{if_T_then_Omega}
	\leanok
	Let $x$ be an integer.
	Every primitive prime divisor of $T_{n}(x)-1$ divides $\C_{n}(x)$.
\end{corollary}

\begin{proposition}\label{T-on-x-plus-one-minus-one-by-x}
	\lean{T_on_x_plus_one_minus_one_by_x}
	\leanok
	For every natural number $n$ and every nonzero real number $x$
	\[\frac{T_{n}(x+1)-1}{x}=n^2+\frac{n^2(n^2-1)}{6}x+\frac{n^2(n^2-1)(n^2-4)}{90}x^{2}+\ldots \in \Z[x],\]
	where the dots denote irrelevant terms.
\end{proposition}

\chapter{Major results on \texorpdfstring{$\C$}{C}'s}

\begin{theorem}\label{Satz-1}
	\lean{Satz_1_part1,Satz_1_part2, Satz_1_part3_ge_5, Satz_1_part3_p_eq_3_corrected,
		Satz_1_part3_p_eq_2_corrected}
	\leanok
	Let $n$ and $a$  be integers such that $n>1$ and $\left| a\right| >1$. If $p$ is a prime number
	dividing $\Cn$ then there exists a nonnegative integer $i$ such that $n=\Che(a)p^{i}$. If $i>0$,
	then $p$ is the greatest prime divisor of $n$. If moreover $p^2 \mid \Cn$ then either $p=2$ and
	$n\in\left\lbrace 2,4\right\rbrace$, or $p=3$ and $n\in\left\lbrace 3,6\right\rbrace $.
\end{theorem}

\begin{corollary}\label{wniosek}
	\lean{wniosek}
	\leanok
	Let $n$ and $a$  be integers such that $n>1$ and $\left| a\right| >1$. The following statements are equivalent.
	\begin{enumerate}[label={\text{S}\arabic*}]
		\item \label{TFAE1} A prime number $p$ such that $n=\Che(a)$ does not exist.
		\item \label{TFAE2} $\left| \Cn \right|$  is a power of a prime number dividing $n$.
		\item \label{TFAE3} $\left| \Cn \right|$  is a power of the greatest prime divisor of $n$.
	\end{enumerate}
\end{corollary}

\end{document}
